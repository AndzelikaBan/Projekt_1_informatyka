\documentclass[a4paper]{article}
\usepackage{graphicx} % Required for inserting images
\usepackage{amsmath,amssymb}
\usepackage[T1]{fontenc}
\usepackage{array}
\setlength{\parindent}{0pt}
\usepackage{hyperref}
\usepackage{pdfpages}
\usepackage[top=2.5cm, bottom=2.5cm, left=2.0cm, right=2.0cm]{geometry}
\usepackage{hyperref}
\usepackage{bibtopic}

\usepackage{geometry}
\usepackage{polski}
\usepackage[utf8]{inputenc}
%\usepackage{fontspec}
%\setmainfont{Times New Roman}

\begin{document}
	
	\newpage
	\begin{titlepage}
		\newgeometry{top = 0.5in,right = 0.5in, left=0.5in, bottom=1in}
		
		\begin{center}
			Politechnika Warszawska \\
			Wydział Geodezji i Kartografii
		\end{center}
		
		\hrule
		\vspace*{1cm}
		\begin{center}
			\Large{\textbf{Informatyka Geodezyjna II}}
		\end{center}
		
		
		\vspace*{2cm}
		\begin{center}
			\large{\textbf{Projekt 1}} 
		\end{center}
		\vspace{3cm}
		\hrule
		
		\begin{center}
			\Large{\textbf{Transformacje.}}
		\end{center}
		\hrule
		
		\vspace*{2cm}
		\begin{center}
			\large{Alicja Dymowska \ \ \ \ 319311} \\
			\large{Nikola Bobik \ \ \ \ 319295} \\
			\large{Andżelika Bańkowska \ \ \ \ 319291}
		\end{center}
		
		\vspace*{3cm}
		
		\begin{center}
			\normalsize{\textbf{Grupa 1}}\\
			\small{Zajęcia: \\poniedziałek 12:15-13:45} \\
			\small{Rok akademicki:\\ 2022/23, Semestr 4}
		\end{center}
		
		\vspace*{3cm}
		\hrule
		\begin{center}
			\large{\textbf{Prowądzacy:} \ mgr. inż. Andrzej Szeszko}
		\end{center}
		\hrule
		
	\end{titlepage}

\newpage
\section{Cel projektu:}
Głównym celem projektu jest stworzenie porgramu, który pozwoli na swobodne przeliczanie współrzędnych na różnych elipsoidach 
\textbf{(GRS80, WGS84, elipsoida Krasowskiego)}. Mamy do wyboru 5 opcji:
	\begin{itemize}
		\item XYZ (geocentryczne) -> BLH (elipsoidalne)
		\item BLH (elipoidalne) -> XYZ (geocentryczne)
		\item XYZ (geocentryczne) -> NEU (topocentryczne)
		\item BL -> PL-1992
		\item BL -> PL-2000
	\end{itemize}

\section{Specyfikacja:}
Program został napisany w języku python (wersja: 3.9), korzystając z bibliotek \textbf{numpy, math, argparse}. Program jest kompatybilny z oprogramowaniem Windows 10 oraz Windows 11. W celu uzyskania instrukcji obsługi programu zapraszamy do zapoznania się z treścią pliku "README.m", który znajduje się na stronie: 
\href{https://github.com/AndzelikaBan/Projekt_1_INF}{GitHub.com}.

\section{Przebieg:}
Do stworzenia programy wykorzystałyśmy funkcje, które definiowałyśmy na przedmiocie Geodezja Wyższa I. Zostały one użyte jako metody pod klasą \textit{Transformacje}. Aby usprawnić przeliczanie między różnymi modelami elipsoid użyłyśmy metodę \textit{init}, dodatkowo zmienne zależne od danych elipsoid zapisalyśmy z użyciem \textit{self}. Z programu można korzystać przy użyciu wiersza poleceń co umożliwiła nam biblioteka \textit{argparse}.  \\
\\
Do poprawnego działania programu będzie niezbędne zainstalowanie bibliotek numpy, math oraz argparse przez użytkownika. Zdecydowałyśmy się na taką opcję, ponieważ biblioteki pozwalają nam na precyzyjniejsze wyniki oraz na schludniejszą formę kodu. 

\section{Utrudnienia:}
Pisząc kod w pythonie miałyśmy problem z wywoływaniem funkcji. Udało nam się zlokalizować źródło problemu, czyli usunęłyśmy z kodu wszystkie polskie znaki. 

\section{Podsumowanie:}
Projekt pozowlił nam na rozwinięcie naszych umiejętności w zakresie programowania w języku python oraz nauczył nowych (napewno w przyłości) przydatnych rzeczy takich jak: \\
\begin{itemize}
	\item Pisanie dokumentów w LaTeX;
	\item Znajomość programowania obiektowego;
	\item Pisanie dokumentacji funkcji - niezbędne dla użytkowników zewnętrznych;
	\item Znajomość portalu GitHub - pozwala na wspólne pisanie kodów, tworząc wspólne repozytoria;
	\item Tworzenie narzędzi w interfejsie tekstowym (cli) potrafiących przyjmować argumenty przy wywołaniu;
	\item Implementowanie algorytmów pochodzących ze źródeł zewnętrznych.
	\end{itemize}

\section{Bibliografia:}
 - \url{http://www.geonet.net.pl/images/2002_12_uklady_wspolrz.pdf}; \\
 - \url{https://notatek.pl/transformacja-wspolrzednych-geocentrycznych-odbiornika-do-wspolrzednych-topocentrycznych}; \\
 - materiały z przedmiotu Geodezja Wyższa I. \\
 \\
\begin{center}
 	\Large{Link do repozytorium: \url{{https://github.com/AndzelikaBan/Projekt_1_INF}}} 
\end{center}
\end{document}